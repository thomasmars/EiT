\chapter{Gruppen}

\textbf{Arnstein D. Svanholm} \\
Arnstein har bachelor i musikkvitenskap, og har derfor hatt hovedansvaret for den musikalske delen ved prosjektet. Kunnskaper innen musikkhistorie har kommet godt med i digitaliseringen av musikkmanuskriptene fra 1700–tallet. Siden manuskriptene er håndskrevet støtte vi bort i en del tolkningsspørsmål i forbindelse med digitaliseringen. I slike tilfeller var kompetanse og erfaring fra komposisjon, musikkanalyse og formkunnskap viktig for å kunne gjøre mest mulig form– og tidsriktige transkripsjoner av manuskriptene. I tillegg til den teoretiske kompetansen har erfaring med bruk av programmet ”Sibelius” vært viktig for digitaliseringsprosessen.\\\\
\textbf{Eino Aleksander Kerr} \\
Aleksander er 24 år og studerer filmvitenskap på NTNU Dragvoll.\\\\
\textbf{Kent Robin Haugen} \\
Kent Robin studerer datateknikk på fjerde året med program og informasjonssystemer som hovedretning. Sammen med Thomas som også studerer datateknikk har de hatt hovedansvaret for programmeringsbiten. Thomas har hatt hovedansvaret for å få til at midisignaler blir sendt ut til mikrokontrolleren som igjen skal lyse opp keyboardet, mens Kent Robin har hatt hovedansvaret for å implementere brukergrensesnittet som Aleksander har designet. 
\\\\
\textbf{Thomas Marstrander} \\
Thomas er XX år og studerer datateknikk på NTNU Gløshaugen.\\\\
\textbf{Vegard Hella} \\
Vegard har bachelorgrad i elektronikk som i hovedsak har basert seg på mikrokontroller løsninger. Den påbegynte mastergraden har tatt en annen retning med vekt på digital signalbehandling og akustikk.\\

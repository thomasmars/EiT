\chapter{Konsept}
\section{Konseptet}
Fra idé til konsept var det en vei å gå. Da gruppen ble enige om hvilken retning ideen skulle ta var det en prosess og få ideen til et konsept. På grunn av tidsbegrensninger og fordi mange i gruppen var i nytt farvann med tanke på hva som skulle utvikles, måtte det naturligvis inngås kompromisser på hvilke funksjoner som skulle være med og ikke være med i den første utgaven av produktet.

Konseptet ble et underholdningsmedium hvor man skulle ha det gøy samtidig som man både bevisst og ubevisst ble introdusert til Trondheims musikkhistorie. Tanken var at systemet skulle stå ferdig montert og klart til bruk hos Gunnerus Bibliotek, så folk som var innom kunne bruke det om de hadde lyst. Systemet som skulle utvikles bestod av følgende komponenter:

\begin{itemize}

\item{Gamle Tangenter}

\item{PC med touchskjerm}

\item{MIDI-keyboard\cite{midikeyboard}}

\item{Lysdioder montert på MIDI-keyboard}

\item{Musikkfiler type MIDI\cite{midi}}

\end{itemize}

\subsection{Gamle Tangenter}
Gamle Tangenter er navnet gruppen har valgt å kalle programmet som kjøres. Denne programvaren er en endret versjon av "open source"-programmet Synthesia 0.6.1\cite{synthesia}. Versjon 0.6.1 av Synthesia var den siste versjonen som fortsatt var "open source"\cite{opensource}, derfor er det denne versjonen Gamle Tangenter bygger på. 

Selve funksjonen til programmet er at det tar inn og tolker musikkfiler av type MIDI, representerer notene som skal spilles i form av rektangulære bokser som flytter seg mot riktig tangent, representert på skjermen. Klarer man å trykke riktig tangent får man poeng og lyden spilles av, treffer man ikke riktig, får man naturlig nok ikke poeng og noten blir stum. Man kan velge om melodilinjene i en sang som skal spilles automatisk av programmet eller om den skal spilles av brukeren. 

Endringene som ble gjort fra Syntheisa 0.6.1 til Gamle Tangenter var en endring i det grafiske grensesnittet. Dette for at Gamle Tangenter skulle være mer i tråd utseendemessig med hva slags type informasjon prosjektet skulle formidle. Fordi musikken dreier seg om gammel musikk fra Trondheim, var det viktig for gruppen å få et grafisk utseende som stod i stil og minnet om noe gammelt.

En annen funksjon som ble lagt til var muligheten for å sende MIDI-signaler fra programmet og ut til lysdiodene på MIDI-keyboardet. 

\subsection{PC med touchskjerm}
Ettersom Gamle Tangenter skal være et ferdig system som skal være overkommelig enkelt å bruke, valgte gruppen og låne en PC med touchskjerm så man skulle slippe å styre programmet med tastatur og mus. Det for å gjøre det enklere og fordi MIDI-keyboardet burde være det eneste input-elementet som burde være synlig og ta opp plass, ettersom hele konseptet baserer seg på input fra denne enheten og dermed burde være hovedfokus. 

\subsection{MIDI-keyboard}


\subsection{Lysdioder montert på MIDI-keyboard}

\subsection{Musikkfiler type MIDI}

\section{Behov}
trengs produktet vårt ...
\section{Hvordan løses problemet}
hvordan løses problemstillinga

\begin{abstract}

Problemstillingen i dette prosjektet gikk ut på hvorvidt det var mulig å få besøkende på Gunnerusbiblioteket i Trondheim, primært fra ungdomsskolen, til å fatte interesse for Trondheims musikkarv, og hvordan dette eventuelt kunne gjennomføres. Problemets forankring i virkelighteten var Gunnerus bibliotek sitt ønske om å få formidlet deres kulturarv på nye, innovative måter. Gruppens tilnærming til problemet var i første omgang å lage en prototype til et produkt som var ment til å stå på Gunnerus biblioteket, og som kunne taes i bruk av de besøkende. Produktet skulle være en pianoversjon av konseptet "Guitar Hero"\cite{guitarhero} hvor man tok i bruk gamle noter og arrangement, komponert for flere hundre år siden og som befant seg i arkivet til Gunnerus bibliotek. Dette konseptet vil sette gammelt notemateriale i et nytt lys, i en ny interaksjonsform, noe som vil engasjere yngre mennesker til denne musikken på en ny måte. Deltakere vil være med å skape musikken på nytt, innenfor et annet medium, og interessen for lokal musikkhistorie vil forhåpentligvis bli revitalisert. 

\end{abstract}

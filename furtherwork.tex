\chapter{Videre Arbeid}
Ettersom det foreligger tidsbegrensninger og at det praktiske ved Eksperter i Team kun er 50 prosent av arbeidet, vil det naturligvis være en hel del forbedringer og videre arbeid forbindet med dette prosjektet. Versjonen av programmet og tangentbelysningen som nå foreligger er kun prototyper som fungerer etter beste evne. 

\section{Programvare}
Programmet som gruppen har brukt som basis i dette prosjektet har flere områder hvor det er forbedringspotensiale. Selve programmet, "Synthesia", som ligger i bunn, fungerer bra som det er, men med tanke på å skreddersy det til vårt formål, er det flere punkter som kan nevnes for å få det til en bedre brukeropplevelse.
\begin{itemize}

\item{Velge sang inne i programmet}
\begin{itemize}
\item{Slik det er nå må man velge sang i et eget vindu som popper opp i det man starter programmet. For at dette skal fungere mer sømløst, hadde det vært bra å kunne sette en egen mappe som standard og at alle sanger legges i denne mappa. Videre da at man kan velge inne på startskjermen i programmet hvilken sang man vil spille utifra de sangene som er i denne mappa.}
\end{itemize}

\item{Maskere valg av input/output-enheter}
\begin{itemize}
\item{Slik det er nå må man spesifisere, før man starter en sang, hvilke input- og outputenheter man skal bruke. Ettersom gruppen foreløpig ikke har noen faste komponenter som skal brukes med dette programmet har ikke dette blitt gjort noe med. Senere når man veit hva slags keyboard, midikontroller og høyttalere som skal brukes permanent med dette programmet, kan man spesifisere dette i koden, og fjerne disse valgene fra programmets brukergrensesnitt. Dette for å gjøre rommet for å gjøre feil hos bruker mindre.}
\end{itemize}

\item{Knapper for å erstatte tastetrykk}
\begin{itemize}
\item{Ettersom programmet kjøres på en touchskjerm, hadde det vært fordelaktig å bytte ut alle kommandoer som kan gjøres med tastaturet med egne knapper i brukergrensesnittet. Dette gjelder spesielt mens man spiller, da muligheten for å endre hastighet på spillet, zooming av noter som spilles, pause, og muligheten for å avslutte påbegynt spill. Nå håndteres alle disse henholdsvis med piltaster og escape.}
\end{itemize}

\end{itemize}

\section{Maskinvare}



\section{Noter}

\section{Design}
\begin{itemize}

\item{Bakgrunnsbilde når man spiller}
\begin{itemize}
\item{}
\end{itemize}

\item{Mer integrert design}
\begin{itemize}
\item{}
\end{itemize}

\item{}
\begin{itemize}
\item{}
\end{itemize}

\end{itemize}